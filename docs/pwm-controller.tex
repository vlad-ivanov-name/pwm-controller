\documentclass[12pt,a4paper]{article}

\renewcommand*{\familydefault}{\sfdefault}

\usepackage{tikz-timing}
\usetikztiminglibrary{nicetabs}

\newcommand{\timingfont}{\fontsize{10pt}{14pt}\selectfont }
\newcommand{\timinglabeloffset}{-.25}

\makeatletter
\long\def\tikztimingtable@row@@#1#2{%
  \addtocounter{tikztiming@nrows}{1}%
  \coordinate (@last row) at ($ (@last row) - (0,\tikztiming@rowdist) $);
  \node [anchor=base east,timing/name,alias=last label] (label\the\c@tikztiming@nrows)
    at ($ (@last row) - (\tikztiming@coldist,\timinglabeloffset) $) {\ignorespaces \timingfont #1\unskip\strut};
  \path let \p1 = (timing@table@bottom left), \p2 = (last label.south west) in
    coordinate (timing@table@bottom left) at ({min(\x1,\x2)},\y2);
  %
  \@ifnextchar{[}%
    {\tikztiming@tabletiming}%
    {\tikztiming@tabletiming[]}%
  #2\relax
  \path let \p1 = (timing@table@bottom right), \p2 = (timing/end base) in
    coordinate (timing@table@bottom right) at ({max(\x1,\x2)},\y2);
  %
  \pgfmathparse{max(\tikztiming@maxwidth,\tikztimingwidth)}%
  \let\tikztiming@maxwidth\pgfmathresult
  \tikztimingtable@checkrow
}
\makeatother

\tikzset{
	timing/yunit = 14pt,
	timing/rowdist = 20pt,
	timing/wscale = 2,
	timing/slope = .25,
	timing/d/text/.append style = {font=\Large}
}

\usepackage{longtable}
\usepackage{tabu}
\renewcommand\arraystretch{1.5}
\tabulinesep=1.5mm

\begin{document}

\section{Introduction}

\section{Control interface}

The device is meant to be controller via SPI control interface. Each SPI transaction consists of two bytes: a command byte and a data byte.

\begin{figure}[h]
\centering
\begin{tikztimingtable}
MISO & 2{u}16{l}16{d}2{u} \\
MOSI & 2{u}16{d}{Command}16{d}2{u} \\
CLK  & 2{u}16{lh}2{u} \\
SS   & 2{h}32{l}2{h} \\
\end{tikztimingtable}
\caption{Generic SPI transaction}
\end{figure}

\begin{figure}[h]
\centering
\begin{tikztimingtable}
MISO & 2{u}16{l}16{d}{Readback}2{u} \\
MOSI & 2{u}2{h}14{d}{Read command}16{h}2{u} \\
CLK  & 2{u}16{lh}2{u} \\
SS   & 2{h}32{l}2{h} \\
\end{tikztimingtable}
\caption{SPI read diagram}
\end{figure}

\begin{figure}[h]
\centering
\begin{tikztimingtable}
MISO & 2{u}16{l}16{u}2{u} \\
MOSI & 2{u}2{l}14{d}{Write command}16{d}{Data}2{u} \\
CLK  & 2{u}16{lh}2{u} \\
SS   & 2{h}32{l}2{h} \\
\end{tikztimingtable}
\caption{SPI write diagram}
\end{figure}

\begin{figure}[h]
\centering
\begin{tikztimingtable}[timing/wscale=4]
MOSI & 2{u}2{d}{R/W}10{d}{Channel}4{d}{Reg}d \\
CLK  & 2{u}8{lh}l \\
SS   & 2{h}16{l}l \\
\end{tikztimingtable}
\caption{Command structure}
\end{figure}

\begin{longtabu} to \textwidth {|X[1]|X[5]|}
\hline
\textbf{Field} & \textbf{Description} \\
\endfirsthead
\hline
\texttt{RW[7]} & Read/write. Set to 1 when performing a read transaction. \\
\hline
\texttt{CH[6:2]} & Channel. Refers to instance of \texttt{pwm} module within the top-level architecture. \\
\hline
\texttt{ADDR[1:0]} & Target register address. \\
\hline
\caption{Command structure}
\end{longtabu}

\end{document}